\documentclass[a4paper; 11pt]{article}
\usepackage[utf8]{inputenc}
\usepackage[english]{babel}
\usepackage{graphicx}
\usepackage{multicol}
\setlength{\columnsep}{1cm}
\bibliographystyle{unsrt}
\begin{document}

\title{Usability of Mobile Map Applications}

\author{Rachel Rivera\\
Professor: Dr. Dionisio\\
CMSI 370: Interaction Design\\
  Loyola Marymount University}

\date{September 25, 2014}

% Create title page with no page number

\renewcommand{\thefootnote}{\fnsymbol{footnote}}

%\singlespacing

\maketitle

\vspace{-.2in}
\begin{center}
\begin{abstract}
\noindent This investigation is motivated by the increasing significance of the usability of systems in today's competitive marketplace. Usability testing and evaluations have become important tools in the assessment of user interfaces. This paper presents empirical mobile usability studies of two distinct map applications: Google Maps and Apple Maps. Participants of this investigation were asked to carry out three tasks with each application. The results of this examination include (a) the contextual factors studied; (b) the core usability metrics measured; (c) quantitative data collected from each task; and (d) a brief qualitative review from each participant. This paper not only presents the usability metrics collected by the study but also presents a heuristic evaluation of each map application.
\end{abstract}
\end{center}

\medskip
\medskip


\medskip
\noindent \textit{Group Members}: Katarina Klask, J.B. Morris, Alex Schneider, Khalid Seirafi, and Ronald Uy.

\thispagestyle{empty}

\clearpage

%\onehalfspacing
\setcounter{footnote}{0}
\renewcommand{\thefootnote}{\arabic{footnote}}
\setcounter{page}{1}

\section{Introduction}
When the Apple Maps mobile application was first released as part of iOS 6 in September 2012, it received a significant amount of negative feedback. Walt Mossberg of the \textit{Wall Street Journal} was one of many who thought that Apple Maps was the biggest drawback of the iPhone 5.\cite{Mossberg} At this time, the Google Maps mobile apps, on both iOS and Android, were used by roughly 81 million people.\cite{ComScore} However, Apple Maps grew, matured, and began gaining traction. By September 2013, Apple Maps gained 35 million regular users. The number of users of Google Maps dropped to around 58.7 million at this time.\cite{ComScore} Although the Apple Maps mobile app is newer than the Google Maps mobile app,\footnote{The initial release of Google Maps for Mobile was in September of 2008.} the two map applications are comparable with respect to features they provide as well as popularity nowadays.
\medskip
\par
Since both Google Maps apps and Apple Maps apps are widely known and used, it is important that the usability metrics of the two systems are examined. Thus, this study aims to empirically investigate the usability of each application. The current consensus within the field of interaction design is that usability depends on five distinct metrics: \textit{learnability}, \textit{efficiency}, \textit{errors}, \textit{memorability}, and \textit{satisfaction}.\cite{Nielsen} The three metrics that this particular study focuses on are efficiency, errors, and satisfaction.
\section{The Model} \label{sec:Model}

Model for testing users. 


\subsection{How to Include Figures}
\begin{figure}[ht]
\centering
\includegraphics[width=0.3\textwidth]{frog.jpg}
\caption{\label{fig:frog}This frog was uploaded to writeLaTeX via the project menu.}
\end{figure}

\subsection{Results}
blah blah blah


\begin{table}[ht]
\centering 
\begin{tabular}{l c c c c c} % The final bracket specifies the number of columns in the table along with left and right borders which are specified using vertical bars (|); each column can be left, right or center-justified using l, r or c. To specify a precise width, use p{width}, e.g. p{5cm}
& \multicolumn{5}{c}{Growth Media} \\ % Amalgamating several columns into one cell is done using the \multicolumn command as seen on this line
Strain & 11 & 2 & 3 & 4 & 5\\ % Column names row
GDS1002 & 0.962 & 0.821 & 0.356 & 0.682 & 0.801\\ % Content row 1
NWN652 & 0.981 & 0.891 & 0.527 & 0.574 & 0.984\\ % Content row 2
PPD234 & 0.915 & 0.936 & 0.491 & 0.276 & 0.965\\ % Content row 3
JSB126 & 0.828 & 0.827 & 0.528 & 0.518 & 0.926\\ % Content row 4
JSB724 & 0.916 & 0.933 & 0.482 & 0.644 & 0.937\\ % Content row 5


Average Rate & 0.920 & 0.882 & 0.477 & 0.539 & 0.923\\ % Summary/total row

\end{tabular}
\caption{Table caption text}
\end{table}

\begin{table}
\centering
\begin{tabular}{l|r}
Item & Quantity \\\hline
Widgets & 42 \\
Gadgets & 13
\end{tabular}
\caption{\label{tab:widgets}An example table.}
\end{table}


\begin{multicols}{2}
[
\section{First Section}
All human things are subject to decay. And when fate summons, Monarchs must obey.
]
Hello, here is some text without a meaning.  This text should show what 
a printed text will look like at this place.
If you read this text, you will get no information.  Really?  Is there 
no information?  Is there...
\end{multicols}
\clearpage
\begin{thebibliography}{100} % 100 is a random guess of the total number of 
%references
\bibitem{Nielsen} Nielsen, Jakob, \emph{Usability Engineering}, Boston: Academic, 1993.
\bibitem{Mossberg} Mossberg, Walt., ``The iPhone Takes to the Big Screen," \emph{The Wall Street Journal}, September 2012.
\bibitem{ComScore}``Analytics for a Digital World - ComScore, Inc." \emph{ComScore,Inc}, 2012.
\bibitem{HK} Kopka, H., Daly P.W., \emph{A Guide to LaTeX},
Addison-Wesley, Reading, MA, 1999.
\bibitem{Pan} Pan, D., ``A Tutorial on MPEG/Audio Compression," \emph{IEEE 
Multimedia}, Vol.2, pp.60-74, Summer 1998.
\end{thebibliography}
%%%%%%%%%%%%% end %%%%%%%%%%%%%%%%%%%%%%%%%%%%%%%



\end{document}