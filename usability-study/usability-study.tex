\documentclass[a4paper, 11pt]{article}

\begin{document}

\title{Usability Metrics for Mobile Map Applications}

\author{Rachel Rivera\\
  Loyola Marymount University}

\date{September 25, 2014}

% Create title page with no page number

\renewcommand{\thefootnote}{\fnsymbol{footnote}}

%\singlespacing

\maketitle

\vspace{-.2in}
\begin{abstract}
\noindent This investigation aimed to record some usability metrics for two mobile map applications as well as explore \textit{why} these applications performed the way that they did. The two map applications examined were Google Maps\footnote{https://maps.google.com} and Apple Maps\footnote{https://www.apple.com/ios/maps}. Users that took part in this study were given iPhone devices and told to execute three concrete tasks with each application; measurements were taken throughout the execution. Usability metrics were then used in order to make a judgment call on which map application performed best.
\end{abstract}

\medskip
\medskip

\noindent \textit{Professor}: Dr. Dionisio

\medskip
\noindent \textit{Group Members}: Katarina Klask, J.B. Morris, Alex Schneider, Khalid Seirafi, and Ronald Uy

\thispagestyle{empty}

\clearpage

%\onehalfspacing
\setcounter{footnote}{0}
\renewcommand{\thefootnote}{\arabic{footnote}}
\setcounter{page}{1}

\section{Introduction}

Interaction design is a field of study that has grown and matured a considerable amount in recent years. The current consensus of the field measures the performance of systems based on five distinct metrics: \textit{learnability}, \textit{efficiency}, \textit{errors}, \textit{memorability}, and \textit{satisfaction}.\footnote[1]{\textit{Vocabulary specifically from Jakob Nielsen}: Nielsen, Jakob. Usability Engineering. Boston: Academic, 1993. Print.} Talkin more about stuff.
\section{The Model} \label{sec:Model}

Model for testing users. 



\subsection{Results}
blah blah blah
\begin{table}[ht]
\centering 
\begin{tabular}{l c c c c c} % The final bracket specifies the number of columns in the table along with left and right borders which are specified using vertical bars (|); each column can be left, right or center-justified using l, r or c. To specify a precise width, use p{width}, e.g. p{5cm}
& \multicolumn{5}{c}{Growth Media} \\ % Amalgamating several columns into one cell is done using the \multicolumn command as seen on this line
Strain & 1 & 2 & 3 & 4 & 5\\ % Column names row
GDS1002 & 0.962 & 0.821 & 0.356 & 0.682 & 0.801\\ % Content row 1
NWN652 & 0.981 & 0.891 & 0.527 & 0.574 & 0.984\\ % Content row 2
PPD234 & 0.915 & 0.936 & 0.491 & 0.276 & 0.965\\ % Content row 3
JSB126 & 0.828 & 0.827 & 0.528 & 0.518 & 0.926\\ % Content row 4
JSB724 & 0.916 & 0.933 & 0.482 & 0.644 & 0.937\\ % Content row 5


Average Rate & 0.920 & 0.882 & 0.477 & 0.539 & 0.923\\ % Summary/total row

\end{tabular}
\caption{Table caption text}
\end{table}


\end{document}