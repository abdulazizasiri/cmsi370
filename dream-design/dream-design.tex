\documentclass[11pt]{article}


\usepackage{geometry}
\geometry{letterpaper}


\usepackage{doc}
\usepackage{cite}
\usepackage[margin=1cm]{caption}

\usepackage{url}


\usepackage{graphicx}
\usepackage{epstopdf}
\DeclareGraphicsRule{.tif}{png}{.png}{`convert #1 `dirname #1`/`basename #1 .tif`.png}


\title{A ``Dream'' Interface Design}
\author{Rachel Rivera}
\date{November 25, 2014}


\begin{document}


\maketitle


\begin{abstract}
This investigation purposes a ``dream'' user interface for touch screen devices. 
\end{abstract}


\pagebreak
\tableofcontents



\pagebreak


\section{Introduction}
\label{Introduction}

Mobile devices are ubiquitous in contemporary society. The latest figures from the United Nations' telecommunications agency estimate that there are around 6.8 billion cell-phone subscriptions in the world \cite{UNTelecommunications	}. Thus, the investigation and development of usable interfaces for mobile devices is a significant research topic.

Many mobile devices utilize touch screens instead of physical keyboards or buttons. This allows for a larger display without increasing the size of the device. Furthermore, touchscreen keyboards and buttons have the ability to change their layout based on user input and disappear when not needed. To compensate for the lack of tactile feedback provided by physical keys and buttons, touchscreen devices often include aural feedback in the form of audible clicks from a speaker. Haptic feedback in the form of device vibrations is often included as well.

Even with these alternate forms of feedback, the literature suggests that insufficient feedback is still a major usability issue with touchscreen mobile devices. \cite{Tinwala:2010:ETE:18	68914.1868972, Kane:2011:UGB:1978942.1979001, Hardy:2008:TIT:1409240.1409267, El-Glaly:2013:TTF:2460625.2460665, Buxton:1986:HID:22339.22386}. A study by Hussain Tinwala and Scott MacKenzie submits that the lack of physical keys requires heightened visual attention from the user, which diverts the user's concentration from the thoughts being expressed \cite{Tinwala:2010:ETE:18 68914.1868972}. The lack of physical keys not only diverts the attention of some users, but it also makes the device  almost entirely unusable for other users. A study from Virginia Polytechnic Institute and State University demonstrates how touchscreen mobile devices do not provide sufficient feedback for Individuals with Blindness or Severe Visual Impairment (IBSVI) as these users are only able to ``develop a spatial mental model for the interface or the screen through dead reckoning'' \cite{El-Glaly:2013:TTF:2460625.2460665}. 

Thus, the aim of this investigation is to propose a ``dream'' interface design that addresses some of these usability issues.


\section{System Description}
\label{System Description}

In this investigation, I propose a ``dream'' interface design for touchscreen mobile devices that focuses on providing the user with as much as feedback as possible. This feedback is designed in a way that aims to be helpful for users with visual impairments and users without visual impairments alike. The design in its entirety was created with the usability metrics of learnability, efficiency, and error rate in mind.

My ``dream'' interface design incorporates functionality of a product that is currently being developed by Tactus Technology \cite{Tactus}. Tactus Technology, a company based in Fremont, California, creates real physical buttons that dynamically appear and disappear into a flat touch screen (see Figure~\ref{tactus1}). Small fluid channels are routed throughout the Tactile Layer and enable fluid to expand the top polymer layer to create the physical buttons \cite{Tactus}.

Although this technology from Tactus is extremely bleeding-edge, the prototypes have already received a fair amount of recognition as well as several awards \cite{CNN, I-Zone, PCMag, Wired}. Reviewers have articulated how the technology seems to be ``downright magical'' \cite{CNN}. Though the look and feel of Tatctus technology seems totally futuristic, the technology is already beginning to appear in consumer devices. In 2013, Touch Revolution, the largest-volume glass projected capacitive multi-touch screen manufacturer in the world, announced a partnership with Tactus Technology \cite{TactusAvailability}. The technologies are being combined and manufactured into consumer devices today \cite{TactusAvailability}. Not only is this technology becoming more and more available, but it is becoming more customizable as well. Companies will soon be able to customize the panel for different types of buttons, say for example, the buttons on a TV remote \cite{CNN}.

\begin{figure}[ht]
\centering
\includegraphics[width=4.5in]{tactus2.jpg} 
\caption{Tactus Technology keyboard}
\label{tactus1}
\end{figure}

My ``dream'' user interface design makes use of the tactile feedback as well as the flexibility that this technology from Tactus provides.


\section{Top-Level Design}
The objective of this ``dream'' interface design is to make mobile touchscreen devices more usable for individuals with visual impairments and individuals without visual impairments alike. The three main features designed to accomplish this goal are the optional physical grid, the tactile buttons of varying sizes, and the auditory feedback. 


\subsection{Optional Physical Grid}
The design incorporates Tactus technology to pa physical grid-like layout on the screen.


\begin{figure}[ht]
\centering
\includegraphics[width=1.5in]{wireframe-grid.png} 
\caption{Example Grid Layout}
\label{wireframe-grid}
\end{figure}

The tactile stuff is intended to operate in conjunction with other technologies that provide auditory feedback that informs the users of \textit{what} is at the location on the screen while the tactile buttons heps the user know \textit{where} it is. 

\subsection{Tactile Buttons}

\subsection{Auditory Feedback}


\section{Usage Scenarios}

\section{Rationale}
The design incorporates Tactus technology to pa physical grid-like layout on the screen. This layout IBSVI so that they can engage their spatial cognition, perception and sensing resources while interacting with touch screens. The design will only enable the grid-like system on-command so to not distract or annoy those who do not benefit from it.

\section{Usability Metric ``Forecast''}
\clearpage


\bibliography{mybib}{}
\bibliographystyle{plain}
\end{document}

\end{document}
	