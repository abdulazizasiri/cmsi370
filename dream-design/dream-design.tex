\documentclass[11pt]{article}


\usepackage{geometry}
\geometry{letterpaper}


\usepackage{doc}
\usepackage{cite}
\usepackage[margin=1cm]{caption}

\usepackage{url}


\usepackage{graphicx}
\usepackage{epstopdf}
\DeclareGraphicsRule{.tif}{png}{.png}{`convert #1 `dirname #1`/`basename #1 .tif`.png}


\title{A ``Dream'' Interface Design}
\author{Rachel Rivera}
\date{November 25, 2014}


\begin{document}


\maketitle


\begin{abstract}
This investigation purposes a ``dream'' user interface for touch screen devices. 
\end{abstract}


\pagebreak
\tableofcontents



\pagebreak


\section{System Description}
\label{System Description}

Mobile devices are ubiquitous in contemporary society. The latest figures from the United Nations' telecommunications agency estimate that there are around 6.8 billion cell-phone subscriptions in the world \cite{UNTelecommunications	}. Thus, the investigation and development of usable interfaces for mobile devices is a significant research topic.

Many mobile devices use touchscreen keyboards instead of physical keyboards. This allows for a larger display without increasing the size of the device. Furthermore, touchscreen keyboards have the ability to change their layout based on user input and disappear when not needed. To compensate for the lack of tactile feedback provided by physical keys, touchscreen keyboards often include aural feedback in the form of audible clicks from a speaker. Haptic feedback in the form of device vibrations is often included as well.

Even with these alternate forms of feedback, the literature suggests that insufficient feedback is still a major usability issue with touchscreen mobile devices. \cite{Tinwala:2010:ETE:18	68914.1868972, Kane:2011:UGB:1978942.1979001, Hardy:2008:TIT:1409240.1409267, El-Glaly:2013:TTF:2460625.2460665, Buxton:1986:HID:22339.22386}. A study by Hussain Tinwala and Scott MacKenzie submits that the lack of physical keys requires heightened visual attention from the user, which diverts the user's concentration from the thoughts being expressed \cite{Tinwala:2010:ETE:18 68914.1868972}. The lack of physical keys not only diverts the attention of some users, but it also makes the device  almost entirely unusable for others. A study from Virginia Polytechnic Institute and State University demonstrates how touchscreen mobile devices do not provide sufficient feedback for Individuals with Blindness or Severe Visual Impairment (IBSVI) as the users are only able to ``develop a spatial mental model for the interface or the screen through dead reckoning'' \cite{El-Glaly:2013:TTF:2460625.2460665}. 

I have designed a ``dream'' interface for touchscreen mobile devices that aims to tackle some of these usability issues. My design provides a significant amount of tactile feedback that will be helpful for individuals without visual impairments and IBSVI alike.




\section{Top-Level Design}
\label{background}

\section{Usage Scenarios}

\section{Rationale}


\section{Usability Metric ``Forecast''}
\clearpage


\bibliography{mybib}{}
\bibliographystyle{plain}
\end{document}

\end{document}
	